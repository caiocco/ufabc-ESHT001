%% abtex2-modelo-artigo.tex, v-1.9.6 laurocesar
%% Copyright 2012-2016 by abnTeX2 group at http://www.abntex.net.br/ 
%%
%% This work may be distributed and/or modified under the
%% conditions of the LaTeX Project Public License, either version 1.3
%% of this license or (at your option) any later version.
%% The latest version of this license is in
%%   http://www.latex-project.org/lppl.txt
%% and version 1.3 or later is part of all distributions of LaTeX
%% version 2005/12/01 or later.
%%
%% This work has the LPPL maintenance status `maintained'.
%% 
%% The Current Maintainer of this work is the abnTeX2 team, led
%% by Lauro César Araujo. Further information are available on 
%% http://www.abntex.net.br/
%%
%% This work consists of the files abntex2-modelo-artigo.tex and
%% abntex2-modelo-references.bib
%%

% ------------------------------------------------------------------------
% ------------------------------------------------------------------------
%  abnTeX2: Modelo de Artigo Acadêmico em conformidade com
%  ABNT NBR 6022:2003: Informação e documentação - Artigo em publicação 
%  periódica científica impressa - Apresentação
% ------------------------------------------------------------------------
% ------------------------------------------------------------------------

\documentclass[
% -- opções da classe memoir --
article,			% indica que é um artigo acadêmico
11pt,				% tamanho da fonte
oneside,			% para impressão apenas no recto. Oposto a twoside
a4paper,			% tamanho do papel. 
% -- opções da classe abntex2 --
%chapter=TITLE,		% títulos de capítulos convertidos em letras maiúsculas
%section=TITLE,		% títulos de seções convertidos em letras maiúsculas
%subsection=TITLE,	% títulos de subseções convertidos em letras maiúsculas
%subsubsection=TITLE % títulos de subsubseções convertidos em letras maiúsculas
% -- opções do pacote babel --
english,			% idioma adicional para hifenização
brazil,				% o último idioma é o principal do documento
sumario=tradicional
]{abntex2}

\setlength\afterchapskip{\lineskip}

% ---
%  PACOTES
% ---

% ---
% Pacotes fundamentais 
% ---
\usepackage[dvipsnames]{xcolor}	% Controle das cores
\usepackage{lmodern}			% Usa a fonte Latin Modern
\usepackage[T1]{fontenc}		% Selecao de codigos de fonte.
\usepackage[utf8]{inputenc}		% Codificacao do documento (conversão automática dos acentos)
\usepackage{indentfirst}		% Indenta o primeiro parágrafo de cada seção.
\usepackage{nomencl} 			% Lista de simbolos
\usepackage{graphicx}			% Inclusão de gráficos
\usepackage{microtype} 			% para melhorias de justificação
\usepackage{glossaries}			% solução de glossário
\usepackage{nameref}			% referenciar também pelo nome
% ---

% ---
%  Pacotes de perfumaria
% ---
\usepackage{lipsum}				% para geração de dummy text
% ---

% ---
%  Pacotes de citações
% ---
\usepackage[brazilian,hyperpageref]{backref}	 % Paginas com as citações na bibl
\usepackage[alf]{abntex2cite}	% Citações padrão ABNT
% ---

% ---
%  Configurações do pacote backref
%  Usado sem a opção hyperpageref de backref
\renewcommand{\backrefpagesname}{Citado na(s) página(s):~}
% Texto padrão antes do número das páginas
\renewcommand{\backref}{}
% Define os textos da citação
\renewcommand*{\backrefalt}[4]{
	\ifcase #1 %
	Nenhuma citação no texto.%
	\or
	Citado na página #2.%
	\else
	Citado #1 vezes nas páginas #2.%
	\fi}%
% ---

% ---
%  Configurações do pacote glossaries
\renewcommand*{\glsclearpage}{}
% ---

% ---
%  Informações de dados para CAPA e FOLHA DE ROSTO
% ---
\titulo{Relatório para a disciplina de  Arranjos Institucionais e Marco Regulatório do Território}
\autor{Caio César Carvalho Ortega}
\local{São Bernardo do Campo, SP}
\data{26/11/2018}
% ---

% ---
%  Configurações de aparência do PDF final e informações do PDF
\makeatletter
\hypersetup{
	%pagebackref=true,
	pdftitle={\@title}, 
	pdfauthor={\@author},
	pdfsubject={Arranjos Institucionais},
	pdfcreator={LaTeX with abnTeX2},
	pdfkeywords={piquiá, vale, parauapebas, mcmv, moradia}, 
	colorlinks=true,		% false: boxed links; true: colored links
	linkcolor=Plum,			% color of internal links
	citecolor=Blue,			% color of links to bibliography
	filecolor=Red,			% color of file links
	urlcolor=Red,
	bookmarksdepth=4
}
\makeatother
% --- 

% ---
%  compila o glossário
% ---
%\makeglossaries

% \newglossaryentry{ex}{name={sample},description={an example}}

% ---

% ---
%  compila o indice
% ---
\makeindex
% ---

% ---
%  Altera as margens padrões
% ---
\setlrmarginsandblock{3cm}{3cm}{*}
\setulmarginsandblock{3cm}{3cm}{*}
\checkandfixthelayout
% ---

% --- 
%  Espaçamentos entre linhas e parágrafos 
% --- 

%  O tamanho do parágrafo é dado por:
\setlength{\parindent}{1.3cm}

%  Controle do espaçamento entre um parágrafo e outro:
\setlength{\parskip}{0.2cm}  % tente também \onelineskip

%  Espaçamento simples
\SingleSpacing

% ----
%  Início do documento
% ----
\begin{document}
	
	% Seleciona o idioma do documento (conforme pacotes do babel)
	%\selectlanguage{english}
	\selectlanguage{brazil}
	
	% Retira espaço extra obsoleto entre as frases.
	\frenchspacing 
	
	% ----------------------------------------------------------
	%  ELEMENTOS PRÉ-TEXTUAIS
	% ----------------------------------------------------------
	
	%---
	%
	% Se desejar escrever o artigo em duas colunas, descomente a linha abaixo
	% e a linha com o texto ``FIM DE ARTIGO EM DUAS COLUNAS''.
	% \twocolumn[    		% INICIO DE ARTIGO EM DUAS COLUNAS
	%
	%---
	% página de titulo
	\maketitle
	
	% ---
	% Título e resumo em língua estrangeira
	% ---
	
	% ----------------------------------------------------------
	%  ELEMENTOS TEXTUAIS
	% ----------------------------------------------------------
	\textual
	
	% ----------------------------------------------------------
	% Introdução
	% ----------------------------------------------------------
	
	\section*{Prólogo}
	\addcontentsline{toc}{section}{Prólogo}
	
	O propósito do presente trabalho é realizar um relatório para a disciplina de Arranjos Institucionais e Marco Regulatório do Território (BH1343).
	
	\section{Relatório}
	
	%	\subsection{Primeiro Relatório}
	
	\noindent \textbf{Data:} 22/10/2018
	
	\noindent \textbf{Tema:} ``os desafios e perspectivas dos moradores e moradoras de Piquiá para superar as violações de direitos provocadas pela indústria de mineração e da siderurgia''
	
	\nocite{palestra2018}
	
	O primeiro a falar é o advogado Danilo Chammas, da ONG Justiça nos Trilhos.
	
	Piquiá, informa o palestrante, é um distrito de Açailândia e a cidade tem cerca de 100 mil habitantes. O distrito é dividido em alto/baixo. São 75 km entre Imperatriz do Maranhão e Açailândia. Imperatriz é a segunda maior cidade do estado. Piquiá de baixo é o núcleo mais afetado.
	
	Chammas exibe mapa do ``Programa Grande Carajás'' e informa que a EFC\footnote{Estrada de Ferro Carajás} tem 892 km. Cita ainda municípios como Parauapebas. Ao todo, são 27 municípios cortados pela estrada de ferro, com 56 pátios.
	
	Exibe mapa "Expansão logística", com o subtítulo ``Projetos da Vale no Norte do país em ferrovia e porto para mineração'', no qual estão destacados investimentos em US\$ 11,4 bi (Projeto Serra Sul) e US\$ 4,1 bi (Projeto CLN150). Em seguida, explica que a EFC terminou de ser duplicada agora em 2018, além disso, também foram feitas ampliações nos portos.
	
	Retomando Piquiá: trata-se de uma das 104 comunidades diretamente impactadas pelo empreendimento da Vale. O pesquisador explica que há mais de 100 grupos humanos, sendo a maioria comunidades de camponeses (organizadas ou não em assentamentos), além de povos tradicionais (como indígenas e quilombolas). Piquiá é uma área urbana.
	
	O advogado e ativista lembra que a Vale nasceu como estatal durante o período do Estado Novo, sendo fundada em Itabira, no estado de Minas Gerais. Na ditadura seguinte a empresa expandiu suas operações. Na década de 1990 foi privatizada durante o governo democrático presidido por Fernando Henrique Cardoso. Relembra que a empresa foi responsável por nosso maior desastre ambiental.
	
	Exibe mapa da Estrada de Ferro Carajás da ANTT\footnote{Agência Nacional de Transportes Terrestres}. Explica que o eixo marcado pela ferrovia na região de Piquiá tem grande diálogo com o agronegócio. Esclarece também que o maior impacto se dá pela operação da ferrovia, incluindo a dificuldade de transposições seguras, o que provoca muitas mortes. Trata-se do maior de carga do mundo: 4 km de extensão e 330 vagões puxados por três locomotivas. Exibe fotografia aérea na qual pode ser vista a ferrovia e seu viaduto sobre o rio, além da rodovia BR-222.
	
	Para contextualizar, Chammas exibe reportagem da TV Mirante (afiliada da Rede Globo no Maranhão). Segundo a reportagem, há apenas 10 passarelas e 14 viadutos. A Vale promete 47 novos viadutos e alega que está investindo em ações de conscientização da comunidade.
	
	O integrante da ONG Justiça nos Trilhos explica que é comum que as pessoas passem por baixo do trem. E a justificativa é que o trem fica muito tempo parado e a população não consegue se programar.
	
	Exibe ainda vídeo amador que documenta protesto em outro município cortado pela ferrovia. É o mesmo município da primeira vítima mostrada na reportagem da TV Mirante. Os passageiros reivindicavam a construção de uma passarela e, para tanto, bloquearam a ferrovia.
	
	A integrante da ONG Usina, a arquiteta Kayla Lazarini, explica que não há política habitacional para casos de reassentamento forçado. Houve recusa do projeto da Vale e opção por uma assessoria própria, que, neste caso, foi a da Usina. Aponta também que há ocorreram acidentes no transporte do ferro líquido e também que há locais com condições análogas à escravidão.
	
	Detalha o projeto, que foi orientado para a vida comunitária e para as pessoas, não para o carro. As casas possuem cerca de 66 m$^{2}$, a cozinha é um local central do desenho da planta, os núcleos contam com pracinhas com um jardim de chuva e a solução de saneamento é local, com bio-valetas, pois mais de 70\% do esgoto da cidade não é tratado.
	
	Sumarização:
	\begin{itemize}
		\item TAC\footnote{Termo de Ajuste de Conduta} para conquistar o terreno;
		\item TAC para conquistar o projeto e assessoria própria;
		\item Amoldamento ao programa de provisão habitacional MCMV;
		\item Complementação de recursos através da Fundação Vale;
		\item Luta pela construção dos equipamentos públicos e institucionais.
	\end{itemize}
	
	Importante ressaltar:
	\begin{itemize}
		\item O perigo da solução via TAC, mantendo a impunidade na essência do problema;
		\item Tendência a judicializar um problema que é político.
	\end{itemize}
	
	Respondendo uma pergunta da plateia, o membro da Justiça Nos Trilhos explica que a EFC originalmente foi construída sem licenciamento ambiental. A duplicação da ferrovia envolveu licenciamento e a questão se encontra judicializada, embora as obras já tenham sido concluídas. No tocante à ferrovia, por se tratar de infraestrutura que ultrapassa os limites de um estado, a fiscalização cabe ao Ibama. No caso das empresas de ferro-gusa, o licenciamento se dá em nível estadual, mas há muitas irregularidades.
	
	%	\subsection{Segundo Relatório}
	
	%	O vídeo está em https://www.youtube.com/watch?v=m7BR9aayZkU
	
	% ---
	% Finaliza a parte no bookmark do PDF, para que se inicie o bookmark na raiz
	% ---
	\bookmarksetup{startatroot}% 
	% ---
	
	% ----------------------------------------------------------
	%  ELEMENTOS PÓS-TEXTUAIS
	% ----------------------------------------------------------
	\postextual
	
	% ----------------------------------------------------------
	% Referências bibliográficas
	% ----------------------------------------------------------
	\bibliography{fontes}
	
	% ----------------------------------------------------------
	% Glossário
	% ----------------------------------------------------------
	% Consultar manual da classe abntex2 para orientações sobre o
	% uso do glossário.
	\renewcommand{\glossaryname}{Glossário}
	%\renewcommand{\glossarypreamble}{Esta é a descrição do glossário.\\ \\}
	\renewcommand*{\glsseeformat}[3][\seename]{\textit{#1}
		\glsseelist{#2}}
	
	% ---
	% Traduções para o ambiente glossaries
	% ---
	\providetranslation{Glossary}{Glossário}
	\providetranslation{Acronyms}{Siglas}
	\providetranslation{Notation (glossaries)}{Notação}
	\providetranslation{Description (glossaries)}{Descrição}
	\providetranslation{Symbol (glossaries)}{Símbolo}
	\providetranslation{Page List (glossaries)}{Lista de Páginas}
	\providetranslation{Symbols (glossaries)}{Símbolos}
	\providetranslation{Numbers (glossaries)}{Números} 
	% ---
	
	% ---
	% Imprime o glossário
	% ---
	%\cleardoublepage
	%\phantomsection
	%\addcontentsline{toc}{section}{\glossaryname}
	%\glossarystyle{index}
	% \glossarystyle{altlisthypergroup}
	% \glossarystyle{tree}
	%\printglossaries
	
	% ----------------------------------------------------------
	% Apêndices
	% ----------------------------------------------------------
	
	% ---
	% Inicia os apêndices
	% ---
	%	\begin{apendicesenv}
	%		
	%		% ----------------------------------------------------------
	%		\chapter{Nullam elementum urna vel imperdiet sodales elit ipsum pharetra ligula
	%			ac pretium ante justo a nulla curabitur tristique arcu eu metus}
	%		% ----------------------------------------------------------
	%		\lipsum[55-57]
	%		
	%	\end{apendicesenv}
	% ---
	
	% ----------------------------------------------------------
	% Anexos
	% ----------------------------------------------------------
	%	\cftinserthook{toc}{AAA}
	% ---
	% Inicia os anexos
	% ---
	%\anexos
	%	\begin{anexosenv}
	%		
	%		% ---
	%		\chapter{Cras non urna sed feugiat cum sociis natoque penatibus et magnis dis
	%			parturient montes nascetur ridiculus mus}
	%		% ---
	%		
	%		\lipsum[31]
	%		
	%	\end{anexosenv}
	
\end{document}
